% !TEX TS-program = pdflatex
% !TEX encoding = UTF-8 Unicode

% This is a simple template for a LaTeX document using the "article" class.
% See "book", "report", "letter" for other types of document.

\documentclass[11pt]{article} % use larger type; default would be 10pt

\usepackage[utf8]{inputenc} % set input encoding (not needed with XeLaTeX)

%%% Examples of Article customizations
% These packages are optional, depending whether you want the features they provide.
% See the LaTeX Companion or other references for full information.

%%% PAGE DIMENSIONS
\usepackage{geometry} % to change the page dimensions
\geometry{a4paper} % or letterpaper (US) or a5paper or....
% \geometry{margin=2in} % for example, change the margins to 2 inches all round
% \geometry{landscape} % set up the page for landscape
%   read geometry.pdf for detailed page layout information

\usepackage{graphicx} % support the \includegraphics command and options

% \usepackage[parfill]{parskip} % Activate to begin paragraphs with an empty line rather than an indent

%%% PACKAGES
\usepackage{booktabs} % for much better looking tables
\usepackage{array} % for better arrays (eg matrices) in maths
\usepackage{paralist} % very flexible & customisable lists (eg. enumerate/itemize, etc.)
\usepackage{verbatim} % adds environment for commenting out blocks of text & for better verbatim
\usepackage{subfig} % make it possible to include more than one captioned figure/table in a single float
% These packages are all incorporated in the memoir class to one degree or another...
% package for affiliations
\usepackage{authblk}

%%% HEADERS & FOOTERS
\usepackage{fancyhdr} % This should be set AFTER setting up the page geometry
\pagestyle{fancy} % options: empty , plain , fancy
\renewcommand{\headrulewidth}{0pt} % customise the layout...
\lhead{}\chead{}\rhead{}
\lfoot{}\cfoot{\thepage}\rfoot{}

%%% SECTION TITLE APPEARANCE
\usepackage{sectsty}
\allsectionsfont{\sffamily\mdseries\upshape} % (See the fntguide.pdf for font help)
% (This matches ConTeXt defaults)

%%% ToC (table of contents) APPEARANCE
\usepackage[nottoc,notlof,notlot]{tocbibind} % Put the bibliography in the ToC
\usepackage[titles,subfigure]{tocloft} % Alter the style of the Table of Contents
\renewcommand{\cftsecfont}{\rmfamily\mdseries\upshape}
\renewcommand{\cftsecpagefont}{\rmfamily\mdseries\upshape} % No bold!

%%% END Article customizations

%%% The "real" document content comes below...

\title{Establishing the radiological inventory of TRIGA IPR-R1 and TRIGA Mainz nuclear reactors aiming the final decommissionig (5 páginas)}
\author[1]{Vitor Vasconcelos Araújo Silva}
\author[2]{Jörg Krämer}
\affil[1]{CDTN}
\affil[2]{Mainz}
\date{} % Activate to display a given date or no date (if empty),
         % otherwise the current date is printed 

\begin{document}
\maketitle

\section{Introduction}

TRIGA reactors are running since 1958...
Many of them were succesfully decommissioned along the years, being the most recent example the FiR-1 TRIGA in Filand, decommissioned in 2014 \cite{Raty_Thesis_2020} following a solid methodology. Unfortunately, due to the very nature of research reactors to have theirs cores modified to accomodate different uses - these reactors are not called reserach reactor for no reason - the detailed history of core configurations are not available in most of the cases \cite[p. xxx]{Clean-up_CEA-e-DEN_2018}. With that in mind, this research project aims to answer that question taking in account the best-practices \cite{AIEA}, the established methodologies \cite{Raty_paper} but also make use of validated numerical simulations to achieve the radiological inventory of a TRIGA reactor in the most accurate level achievable.

\section{Objectives}

The objective of this work is establish the radiological inventory of all components of the TRIGA IPR-R1 and TRIGA Mainz reactors in order to make viable the decommissioing process of these reactors.

\textbf{NOTAS conversa com André: O edital pede alguma ligação com a experiência prévia do pesquisador. No meu caso, acoplamento. Pra isso, a ideia é fazer experimentos no TRIGA com os combustíveis instrumentados e, com os perfis levantados, poder fazer a validação do núcleo simulado por eles. \textit{Com a validação dos cálculos neutrônicos, é possível estabelecer que aqueles são os fluxos e, talvez, definir como fazer o inventário radiológico.}}

In order to show eventual influence of pulse operation mode on the fuel burning, it is important to take into account the less important contribution of delayed neutrons to the neutron flux as the amount U\textsuperscript{235} decreases as shown in \cite{Carvalho2024} and \cite{Carvalho2025} for mixed oxide fuels.

\subsection{Specific objectives}

The specific objectives can be presented as a list:
\begin{itemize}
  %\item Simulate the core of the two reactors by use of Monte Carlo based codes to establish the neutron historical neutron fluxes;
  \item  Experimental measures of obtained with the use of instrumented fuel elements in order to obtain fuel temperatures and, inderectly, cladding and water temperatures in order to serve as reference for neutronics-thermal-hydraulics numerical simulations;
  \item Simulate the core of TRIGA Mainz reactor by use of Monte Carlo based code with the expected/calculated current fuel mixture coupled with thermal-hydraulics CFD simulation to have an accurate feedback of water temperature in the neutronics. 
\item In parallel, use the neutron flux to calculate activation of important materials of the reactors that might have become of radiological importance to the radiological inventory;
\item Deplete the fission materials in order to obtain the most realistic neutron flux based on historical use of the reactors;
\item Validate the calculations with the experimental results
  \item Calculate the shielding properties of a tentative storage cask for TRIGA reactors.
  
\end{itemize}

For the coupling calculations the code to be used is \cite{OpenMC2015} and for radiological measurements the package \cite{SCALE_632}. OpenMC is a opensource software whereas SCALE has a different license but is already available in the Universität Johannes Gutenberg Mainz.

\section{Outcome}

Considering a periodo of 18 months, we expected to have one paper published and another submitted by the end of the collaboration. The first paper should present the results of the two cores - both TRIGA IPR-R1 and TRIGA Mainz - validated using two different Monte Calo codes, namely \cite[OpenMC]{OpenMC2015} and \cite[SCALE]{SCALE_632}. A second paper - and the main objective of the colaboraion proposed in this very document - is, based on the data obtained on the previous simulations, to depleted and burn simulated fissile material in order to establish for both reactors the current radiological inventory. These inventories, as explained earlier, are key to performe most of the tasks expected in a nuclear decommissioing process.

\textbf{IMPORTANT}

\textbf{Simulate only flux for both reactors might not be enough to have a paper published in a good journal. Perhaps, o more realistic expectation is to have the inventory as a first result and the shielding and its related tasks/simulations to a second paper. Two subsmissions in a 18 months period seems more realistic.}

\section{Methodology}

\section{Citations}

A citation is the CEA publication \cite{Clean-up_CEA-e-DEN_2018}, that should be read from pages $5--10$, $13--33$ and $121--123$.

Another citation is another CEA publication \cite{Neutronics_CEA-e-DEN_2015}. Pages still do be chosen.

Probably the most important document to be referenced in this project is Dr. Antti R{\"a}ty thesis \cite{Raty_Thesis_2020}. 

An important document from IAEA is a Technical Report Series \cite{IAEA_TRS_494_2024}.

\subsection{Varios notes}

In this section, I will keep itemized notes that I expect to help when writing the first draft

\begin{enumerate}
\item Para qualquer resultado válido do inventário readiológico, é importante ter o histograma de potência (ou a história de potência) do reator \cite[p.~21]{Clean-up_CEA-e-DEN_2018}. No caso, posso começar o projeto com o histórico até 1987 (dados que tenho). Isso seria justificado pela presunção de que é suficiente para arrancar com a segunda parte da metodologia e não atrasar o trabalho conjunto, já que o histórico pode ser adicionado depois com simulações complementares.
\item Cálculos complexos exigem cálculos de nêutrons \textit{gamma} e \textit{fótons}.\cite[p.~22]{Clean-up_CEA-e-DEN_2018} com geometrias 3D.
\item Impurezas não são levadas em conta, geralmente, em cálculos de transporte de nêutrons. Entretanto, essas têm grande importância no cálculos do termo-fonte \cite[p.~24]{Clean-up_CEA-e-DEN_2018}.
  \item \textbf{Importante:} É fundamental que haja uma seção tratando de como serão tratadas as diversas incertezas dos cálculos, ao menos projetando os problemas. \cite[p.~165-170]{Neutronics_CEA-e-DEN_2015}.
\end{enumerate}

\bibliographystyle{plain} % We choose the "plain" reference style
\bibliography{bibliography}

\end{document}
